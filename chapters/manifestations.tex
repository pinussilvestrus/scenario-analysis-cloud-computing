\section{Darstellung von Ausprägungen}
\label{manifestations}

Die zweite Phase der Szenarioanalyse bildet die Erstellung von Ausprägungen \cite{spath}. Hierfür werden Entwicklungspfade aus denen im vorhergehenden Kapitel definierten Schlüsselfaktoren gebildet. Im Folgenden wird somit für jeden Schlüsselfaktor eine sogenannte Trendprojektion bis hin zu einem bestimmten Punkt in der Zukunft erstellt \cite{mietzner}. Dieser wird grundsätzlich bis zu einer Spanne von 10 Jahren gesehen, da das Cloud Computing in seiner Entwicklung noch relativ am Anfang steht und eine zu weit entfernte Projektion sehr vage wäre. Als validierende Grundlage hierfür bildeten Experteninterviews sowie eine vertiefte Literaturrecherche.

Die vorangestellte Literaturrecherche zur Ermittlung der Einflussfaktoren haben erwiesen, dass es für jeden einzelnen zwei unterschiedliche Entwicklungspfade gibt. Dies sind zum einen ein positiver Pfad, in welchem die Vorteile des Cloud Computing überwiegen. Zum anderen existieren eine Reihe von Risiken für jeden Schlüsselfaktor, welche einen erfolgreichen Einsatz der Technologie verhindern können. Demzufolge liegt es nah, nachfolgend für jeden Schlüsselfaktor ein \textit{Best-Case-} und eine \textit{Worst-Case-Projektion} zu erstellen.

Bei dem Faktor \textit{Informationssicherheit} lässt sich ein Entwicklungspfad für eine sichere Cloud abbilden. Die in die Cloud gestellten Daten wären \textit{Safe in the sky }\cite{almajalid}, wodurch eine komplett sichere Lernumgebung geschaffen werden kann, ohne Bedanken von Datendiebstahl \cite{meinel} \cite{renz}. Im Gegensatz dazu steht die Ausprägung der \textit{Flächendeckenden Systemangriffe}. Die Cloud wäre durch eine schwache Sicherheitsinfrastruktur Ziel für immer wiederkehrende Angriffe auf vertrauliche Daten \cite{gebauer}. Gerade im Kontext von Großunternehmen wäre dies ein großes Risiko im Bezug der vielen vorhandenen Kundendaten \cite{stute}.

Der Best-Case beim Faktor \textit{Störungssicherheit} kennzeichnet ein durchgängig \textit{stabiles Netzwerk} der angebotenen Dienste. Systemeinbrüche und Informationsverluste sind Seltenheiten, Back-Ups von Kundenseite müssen kaum oder gar nicht vorgenommen werden \cite{almajalid}. Der Worst-Case tritt dann ein, wenn dauerhafte Störungen auftreten, hervorgerufen durch nicht ausreichende Bereitstellung von Netzwerkbandbreite \cite{gebauer}. Man erhält ein \textit{instabiles Netzwerk}, gekennzeichnet durch Schwankungen im System, eine hohe Zahl von Ausfällen und zu langen Ladezeiten bei der Bearbeitung von Prozessen \cite{gebauer}.

Eine positive Ausprägung beim Faktor \textit{Skalierbarkeit} wäre durch eine hohe Flexibilität in der Gestaltung der angebotenen Cloud-Dienste spürbar. Die Leistungsfähigkeit würde mit den Kundenanforderungen mitwachsen, beispielsweise durch Resource Pooling oder einer Mischung von Hybrid und Private Clouds \cite{renz}. Dadurch erhält man eine \textit{mitwachsende Cloud}, die große Lernwelten erschaffen kann \cite{almajalid}. Im Gegenzug dazu steht eine \textit{wachstumsverhindernde Cloud}. Diese kennzeichnet eine begrenzte Adaption der angebotenen Dienste an aktuelle Leistungsanforderungen und verhindert die flexible Gestaltung von Speicherplatz und Rechenleistung \cite{gebauer}.

Der Faktor \textit{zuverlässige und schnelle Datenübertragung} würde seinen Best-Case in eine eben solche \textit{zuverlässigen und schnellen Cloud} finden. Der Kunde würde hierbei nicht das Gefühl erhalten, dass sich seine Daten weit weg im Rechenzentrum, sondern gefühlt direkt im Unternehmen befinden \cite{almajalid}. Dies würde zu einem schrankenlosen Zugang zu Bildungsangeboten führen \cite{meinel}. Der Worst-Case in diesem Fall wäre eine \textit{langsame Cloud}. Diese kennzeichnet langsame Übertragungsraten und führt zu einer Verhinderung von produktiver Arbeit \cite{gebauer}.

On-Demand Services sind das Stichwort bei der positiven Ausprägung der \textit{Kontrollmöglichkeiten}. Der Nutzer kann hierbei automatisch seine Infrastruktur erweitern, verkleinern bzw. seinen Wünschen anpassen \cite{stieninger}. Logisch wäre hierbei ein einfacher Wechsel zwischen Public, Private und Hybrid Cloud \cite{alabbadi}. Man erhält eine \textit{individuelle Cloud}. Im Gegenzug dazu steht eine \textit{einseitige Cloud}. Der Kunde erhält ein nur starres Bezahl - und Dienstleistungsmodell \cite{stieninger}. Dies führt zu starken Abhängigkeiten zum Anbieter und zu anderen Kunden \cite{gebauer}.

Eine positive Ausprägung der \textit{Orts - und Zeitunabhängigkeit} kennzeichnet die Möglichkeit des mobilen Lernens \cite{specht}. Zudem ebnet es die Chancen für einen mobilen Arbeitsplatz und einen schrankenlosen Zugang zu Bildungsangeboten \cite{meinel}. Man erhält eine \textit{unabhängige Cloud}. Im Gegensatz dazu steht eine \textit{nur scheinbare Cloud}, die nur vorgibt, die eigentlichen Vorteile der Cloud Technologie widerzuspiegeln. Sie arbeitet in Wirklichkeit nur lokal und man erhält keine Möglichkeit des mobilen Arbeitens. Der Image-Faktor wird hier zweckentfremdet \cite{gebauer}.

Ein positiver Beitrag der \textit{Vernetzung von organisatorischen und prozessualen Strukturen} kann zum Beispiel in der Standardisierung von Prozessen liegen \cite{krcmar}, wodurch eine Erhöhung der Effizienz von Geschäftsprozessen durch Netzwerke erzielt werden kann \cite{schweizer}. Hinzukommt das Potenzial von MOOCs und Social Learning \cite{goertz}, was dahingehend eine \textit{vernetzte Cloud} darstellt. Im negativen Falle kann eine Cloud aber auch \textit{störend} wirken, indem sie interne Prozesse stört und das Zusammenspiel von Unternehmenskomponenten aktiv durch Netzwerkstörungen verhindert \cite{gebauer}.