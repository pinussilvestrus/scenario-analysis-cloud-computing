\section{Identifikation von Zukunftstrends}
\label{futuretrends}

Eine sehr wichtiges Indiz für die Entwicklungsfähigkeit einer Technologie bilden sogenannte Zukunftstrends. Zudem ist die Kenntnis von zukünftigen Wettbewerbsvorteilen aus eben jenen Zukunftstrends ein wichtiges Mittel in der strategischen Unternehmensplanung \cite{mietzner}. Deshalb liegt es nahe, diese im Folgenden zu erörtern. Die Identifikation wird auf Basis von übergeordneten, globalen  Trends und branchenrelevanten Trends durchgeführt.

\subsection{Eingrenzung von Megatrends}

Maßgebend für die Bestimmung von Zukunftstrends sind die sogenannten \textit{Megatrends} \cite{zpunkt}. Durch seine vernetzenden Eigenschaften lässt sich das Cloud Computing in eine Vielzahl dieser Megatrends einordnen:

\begin{itemize}
	\item 02 - Neue Stufe der Individualisierung
	\item 07 - Digitale Kultur
	\item 09 - Ubiquitäre Intelligenz
	\item 12 - Wissensbasierte Ökonomie
	\item 14 - Wandel der Arbeitswelt 
\end{itemize}

Grundsätzlich ist das Cloud Computing in den neunten Megatrend \textit{Ubiquitäre Intelligenz} einzuordnen, welches klar nach dem Cloud-Paradigma definiert ist \cite{zpunkt}. Jedoch können die genannten Vorteile in vielerlei Hinsicht eingesetzt werden und haben somit einen großen Einfluss in die anderen genannten Megatrends.

\subsection{Branchenrelevante Zukunftstrends}

Der Begriff des \textit{E-Learnings} ist schon derzeit allgegenwärtig. Neben dem Lernen mit dem klassischen Lehrbuch etabliert sich zunehmend der Einsatz von digitalen Medien \cite{meinel2}. Nach einer Delphi-Studie von Goertz et al. aus dem Jahre 2013 wird dem sogenannten \textit{Blended Learning}, also einer Synergie zwischen klassischem Präsenzunterricht und digitalem Lernen, mit 99\% eine enorm hohe Rolle in der Lehre zugeordnet \cite{goertz}.

In diesem Zusammenhang steht der Trend des \textit{mobilen Lernens}. Lernende sollen die Möglichkeit erhalten, auch außerhalb des Unterrichts unabhängig vom Arbeitsplatz auf Lehrmaterialien zugreifen zu können. Nach Specht et al. soll das Lernen dadurch zunehmend individueller werden \cite{specht}. Daraus ergeben sich natürlich auch Verbindungsoptionen mit anderen Technologiefeldern. Dazu gehören u.a. ortsbasierte und kontextsensitive Lerntechnologien mittels der Erfassung geotechnologischer Daten, Augmented Reality beim Einsatz von virtuellen Lernspielen sowie der Einsatz von künstlicher Intelligenz in interaktiven Lernanwendungen.

Der Trend geht somit klar in Richtung einer vernetzten Lernumgebung. Die in Abschnitt \ref{environment} dargestellten Eigenschaften des Cloud Computing verstärken diese Tendenz ganz klar. Somit ist nachfolgend Interessant, Einflussfaktoren für den erfolgreichen Einsatz dieser Technologie zu ermitteln.