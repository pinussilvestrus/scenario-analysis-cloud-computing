\section{Zusammenfassung und Handlungsempfehlungen}
\label{conclusion}
Aus den erarbeiten Zukunftsszenarien ergeben sich nun Handlungsempfehlungen für die fiktive Skola GmbH. Diese können dementsprechend auf die tatsächliche Situation in der Bildungsbranche angewendet werden.

Die in Abschnitt \ref{constructions} gezeigten Szenarien haben gezeigt, dass ein Einsatz von Cloud Services durch das Unternehmen durchaus sinnvoll ist (vgl. Szenario 1 und 2). Dies kann auf unterschiedlichem Wege passieren. Als Schulbuchhersteller kann die Skola GmbH auf bereits vorhandene Ressourcen in Form ihrer Schulbücher sowie Arbeitsmaterialien zurückgreifen und diese beispielsweise als eigenständiger Cloud Provider im Internet anbieten. Dadurch kann man auf die Zukunftstrends \textit{mobiles Lernen}, \textit{E-Learning} oder \textit{Blended Learning} aufspringen. Man kann hierbei in alternative Richtungen gehen. Zum Beispiel könnte man im Bezahlmodell oder in der Art der angebotenen Dienste variieren (vgl. Abschnitt \ref{environment}). Dazu ist zwingend technisches Know-How notwendig. 

Will man als eigenständiger Provider auftreten, muss man zudem die Risiken der Cloud Technologie berücksichtigen (vgl. Szenario 3). Eine Reduzierung der Probleme kann zum Beispiel durch eine Verlagerung dieser Dienste auf externe Cloud Provider vorgenommen werden, wie zum Beispiel \textit{Amazon Web Services} oder \textit{Google Cloud Platform}. Dadurch liegt ein Großteil der Verantwortung bei den externen Anbietern und man kann sich auf die Bereitstellung der Lernmaterialien konzentrieren.

Wichtig in diesem Kontext ist zudem die Problematik des Datenschutzes. Geht man zum Beispiel den Weg über externe Dienstleister, muss beachtet werden, dass diese gegenwärtigen Richtlinien entsprechen. Gerade bei ausländischen Anbietern ist dies momentan schwer oder gar nicht gewährleistet. Kritisch in diesem Zusammenhang ist zudem der vorhandene Bildungsförderalismus in der deutschen Lernindustrie.

Grundsätzlich lässt sich jedoch zusammenfassen, dass das Cloud Computing enorme Potentiale im Bildungsbereich bietet. Es ist zunehmend wichtig, die Vorteile der Technologie zu bündeln, um diese Potentiale ausschöpfen zu können. Essentiell ist hierbei ein bundesweit einseitiges, datenschutzrechtliches ansprechendes Konzept.