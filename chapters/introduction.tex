\section{Einleitung}
\label{introduction}
Das Gebiet des Cloud Computing gehört zu den am meist verwendeten \textit{Buzz-Words} der gegenwärtigen, industriellen Digitalisierung. \todo{statistik!} Hinter einem weit gedehnten Begriff birgt sich eine Vielzahl von Potentialen für neue Geschäftsmodelle. Vordergründig die Bildungsbranche kann durch einen gezielten Einsatz von Cloud-Technologien neue Innovationspotenziale ausschöpfen. \todo{Zitat?} Das Ziel der vorliegenden Arbeiten ist es somit, mithilfe eine technologischen Szenarioanalyse diese Potentiale zu identifizieren. Es sollen mögliche Einsatzzwecke aufgezeigt und darauf gebildete Entwicklungsszenarien für die Bildungsbranche erarbeitet werden.

\todo[inline]{todo:}
-Beschreibung Vorhaben \\
-Kurz-Beschreibung Firma \\
-Beschreibung Vorgehen (Szenarioanalyse 3 Phasen, Übersicht der Kapitel), nach \cite{spath} und \cite{mietzner}

Die erste Phase der Szenarioanalyse bildet eine umfassende Darstellung des Szenarioumfelds  \cite{spath}. Hierfür werden zunächst branchen - sowie technologiebezogene Betrachtungsschwerpunkte definiert (Kapitel \ref{environment}). Nachfolgend werden in den Abschnitten \ref{futuretrends} und \ref{influencingfactors} relevante Zukunftstrends sowie Einflussfaktor im Bezug des Cloud Computing dargestellt. Mithilfe eines Faktorenportfolios und einer Einflussmatrix werden zudem kritische Einflussfaktoren hervorgehoben. Auf dessen Grundlage entstehen in Kapitel \ref{manifestations} Zukunftsausprägungen, was die zweite Phase der Analyse darstellt. Anschließend werden drei wesentliche Szenarien aus den vorher entstandenen Erkenntnissen konstruiert (Abschnitt \ref{constructions}), woraus sich Handlungsempfehlungen und Konsequenzen für das dargestellte Unternehmen ergeben (Abschnitt \ref{conclusion}).