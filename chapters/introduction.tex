\section{Einleitung}
\label{introduction}
Das Gebiet des Cloud Computing gehört zu den am meist verwendeten \textit{Buzz-Words} der gegenwärtigen, industriellen Digitalisierung. Laut dem ''Cloud Monitoring'' - Bericht von 2017 nutzen bereits ca. 65\% diese Technologie \cite{bitkom}. Tendenz ganz klar steigend. Hinter einem weit gedehnten Begriff birgt sich eine Vielzahl von Potentialen für neue Geschäftsmodelle. Vordergründig die Bildungsbranche kann durch einen gezielten Einsatz von Cloud-Technologien neue Innovationspotenziale ausschöpfen \cite{meinel2}. Das Ziel der vorliegenden Arbeiten ist es somit, mithilfe eine technologischen Szenarioanalyse diese Potentiale zu identifizieren. Es sollen mögliche Einsatzzwecke aufgezeigt und daraus Entwicklungsszenarien für die Bildungsbranche erarbeitet werden.

Die folgenden Ergebnisse sollen auf die fiktive \textit{Skola GmbH} bezogen werden. Das Unternehmen ist als Schulbuchverlag seit über 40 Jahren in der deutschen Bildungsbranche aktiv. Derzeit erfolgt jedoch der Vertrieb immer noch vollständig über nichtdigitalisierte Schulbücher, ohne jegliche Verfügbarkeit im Internet. Im Zuge dessen hat sich in den letzten Jahren eine klarer Wettbewerbsnachteil gebildet. Die vorliegende Arbeit soll somit Wege aufzeigen, mithilfe von Cloud-Technologien neue Vorteile am hart umkämpften Markt zu generieren. Dafür wird eine Szenarioanalyse nach Spath et al. \cite{spath} und Mietzner \cite{mietzner} durchgeführt.

Die erste Phase der Szenarioanalyse bildet eine umfassende Darstellung des Szenarioumfelds  \cite{spath}. Hierfür werden zunächst branchen - sowie technologiebezogene Betrachtungsschwerpunkte definiert (Kapitel \ref{environment}). Nachfolgend werden in den Abschnitten \ref{futuretrends} und \ref{influencingfactors} relevante Zukunftstrends sowie Einflussfaktoren im Umfeld des Cloud Computing dargestellt. Mithilfe eines Faktorenportfolios und einer Einflussmatrix werden zudem kritische Einflussfaktoren hervorgehoben. Auf dessen Grundlage entstehen in Kapitel \ref{manifestations} Zukunftsausprägungen, was die zweite Phase der Analyse darstellt. Anschließend werden drei wesentliche Szenarien aus den vorher entstandenen Erkenntnissen konstruiert (Abschnitt \ref{constructions}), woraus sich Handlungsempfehlungen und Konsequenzen für das dargestellte Unternehmen ergeben (Abschnitt \ref{conclusion}).