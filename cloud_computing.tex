\documentclass[conference]{IEEEtran}
\IEEEoverridecommandlockouts
% The preceding line is only needed to identify funding in the first footnote. If that is unneeded, please comment it out.
\usepackage[noadjust]{cite}
\usepackage[utf8]{inputenc}
\usepackage[T1]{fontenc}
\usepackage[ngerman]{babel}
\usepackage{cite}
\usepackage{amsmath,amssymb,amsfonts}
\usepackage{algorithmic}
\usepackage{graphicx}
\usepackage{float}
\usepackage{placeins}
\usepackage{textcomp}
\usepackage{multicol}
\usepackage[colorinlistoftodos,prependcaption,textsize=tiny]{todonotes}
\def\BibTeX{{\rm B\kern-.05em{\sc i\kern-.025em b}\kern-.08em
    T\kern-.1667em\lower.7ex\hbox{E}\kern-.125emX}}

%override ieee citation style
\renewcommand{\citepunct}{,\penalty\citepunctpenalty\,}
\renewcommand{\citedash}{--}% optionally
\begin{document}

\selectlanguage{ngerman}
\title{Technologische Szenarioanalyse für den Einsatz von Cloud Computing in der Bildungsbranche}

\author{\IEEEauthorblockN{Niklas Kiefer}
\IEEEauthorblockA{\textit{Hochschule Harz} \\
Wernigerode, Deutschland \\
u33505@hs-harz.de}}

\maketitle

\begin{abstract}
.. todo ..
\end{abstract}

\begin{IEEEkeywords}
technological scenario analysis, cloud computing, education industry
\end{IEEEkeywords}

\section{Einleitung}
\label{introduction}
Das Gebiet des Cloud Computing gehört zu den am meist verwendeten \textit{Buzz-Words} der gegenwärtigen, industriellen Digitalisierung. Laut dem ''Cloud Monitoring'' - Bericht von 2017 nutzen bereits ca. 65\% diese Technologie \cite{bitkom}. Tendenz ganz klar steigend. Hinter einem weit gedehnten Begriff birgt sich eine Vielzahl von Potentialen für neue Geschäftsmodelle. Vordergründig die Bildungsbranche kann durch einen gezielten Einsatz von Cloud-Technologien neue Innovationspotenziale ausschöpfen \cite{meinel2}. Das Ziel der vorliegenden Arbeiten ist es somit, mithilfe eine technologischen Szenarioanalyse diese Potentiale zu identifizieren. Es sollen mögliche Einsatzzwecke aufgezeigt und daraus Entwicklungsszenarien für die Bildungsbranche erarbeitet werden.

Die folgenden Ergebnisse sollen auf die fiktive \textit{Skola GmbH} bezogen werden. Das Unternehmen ist als Schulbuchverlag seit über 40 Jahren in der deutschen Bildungsbranche aktiv. Derzeit erfolgt jedoch der Vertrieb immer noch vollständig über nichtdigitalisierte Schulbücher, ohne jegliche Verfügbarkeit im Internet. Im Zuge dessen hat sich in den letzten Jahren eine klarer Wettbewerbsnachteil gebildet. Die vorliegende Arbeit soll somit Wege aufzeigen, mithilfe von Cloud-Technologien neue Vorteile am hart umkämpften Markt zu generieren. Dafür wird eine Szenarioanalyse nach Spath et al. \cite{spath} und Mietzner \cite{mietzner} durchgeführt.

Die erste Phase der Szenarioanalyse bildet eine umfassende Darstellung des Szenarioumfelds  \cite{spath}. Hierfür werden zunächst branchen - sowie technologiebezogene Betrachtungsschwerpunkte definiert (Kapitel \ref{environment}). Nachfolgend werden in den Abschnitten \ref{futuretrends} und \ref{influencingfactors} relevante Zukunftstrends sowie Einflussfaktoren im Umfeld des Cloud Computing dargestellt. Mithilfe eines Faktorenportfolios und einer Einflussmatrix werden zudem kritische Einflussfaktoren hervorgehoben. Auf dessen Grundlage entstehen in Kapitel \ref{manifestations} Zukunftsausprägungen, was die zweite Phase der Analyse darstellt. Anschließend werden drei wesentliche Szenarien aus den vorher entstandenen Erkenntnissen konstruiert (Abschnitt \ref{constructions}), woraus sich Handlungsempfehlungen und Konsequenzen für das dargestellte Unternehmen ergeben (Abschnitt \ref{conclusion}). % Kapitel 1: Einleitung
\section{Definition des Szenarioumfelds}
\label{environment}

Gegenstand des folgenden Kapitels soll es sein, relevante Umweltfaktoren herauszuheben. Dies geschieht  im Branchen- sowie im Technologiekontext des Untersuchungsobjekt Cloud Computing. 

\todo[inline]{Unternehmensumfeld (Bildungsbranche, grundsätzlich global
	Schulbuchverlage, Definition Cornelsen
	Wandel zum digitalen Lernen: Angebot der Bücher und Arbeitsmaterialien webbasiert=
	Technisches Umfeld Cloud Computing (
	Definition etc.
	Abgrenzung Private-, Public-, Community- und Hybrid-Cloud
	Abgrenzung SaaS, IaaS, PaaS)
	}

\begin{figure}
	\centering
	\includegraphics[width=\linewidth]{images/bigpicture}
	\caption[Caption for parameters]{ "Big Picture" des Cloud Computing nach Stieninger \cite{stieninger}}
	\label{fig:bigpicture}
\end{figure}  % Kapitel 2: Definition des Szenarioumfelds
\section{Identifikation von Zukunftstrends}
\label{futuretrends}

Eine sehr wichtiges Indiz für die Entwicklungsfähigkeit einer Technologie bilden sogenannte Zukunftstrends. Zudem ist die Kenntnis von zukünftigen Wettbewerbsvorteilen aus eben jenen Zukunftstrends ein wichtiges Mittel in der strategischen Unternehmensplanung \cite{mietzner}. Deshalb liegt es nahe, diese im Folgenden zu erörtern. Die Identifikation wird auf Basis von übergeordneten, globalen  Trends und branchenrelevanten Trends durchgeführt.

\subsection{Eingrenzung von Megatrends}

Maßgebend für die Bestimmung von Zukunftstrends sind die sogenannten \textit{Megatrends} \cite{zpunkt}. Durch seine vernetzenden Eigenschaften lässt sich das Cloud Computing in eine Vielzahl dieser Megatrends einordnen:

\begin{itemize}
	\item 02 - Neue Stufe der Individualisierung
	\item 07 - Digitale Kultur
	\item 09 - Ubiquitäre Intelligenz
	\item 12 - Wissensbasierte Ökonomie
	\item 14 - Wandel der Arbeitswelt 
\end{itemize}

Grundsätzlich ist das Cloud Computing in den neunten Megatrend \textit{Ubiquitäre Intelligenz} einzuordnen, welches klar nach dem Cloud-Paradigma definiert ist \cite{zpunkt}. Jedoch können die genannten Vorteile in vielerlei Hinsicht eingesetzt werden und haben somit einen großen Einfluss in die anderen genannten Megatrends.

\subsection{Branchenrelevante Zukunftstrends}

Der Begriff des \textit{E-Learnings} ist schon derzeit allgegenwärtig. Neben dem Lernen mit dem klassischen Lehrbuch etabliert sich zunehmend der Einsatz von digitalen Medien \cite{meinel2}. Nach einer Delphi-Studie von Goertz et al. aus dem Jahre 2013 wird dem sogenannten \textit{Blended Learning}, also einer Synergie zwischen klassischem Präsenzunterricht und digitalem Lernen, mit 99\% eine enorm hohe Rolle in der Lehre zugeordnet \cite{goertz}.

In diesem Zusammenhang steht der Trend des \textit{mobilen Lernens}. Lernende sollen die Möglichkeit erhalten, auch außerhalb des Unterrichts unabhängig vom Arbeitsplatz auf Lehrmaterialien zugreifen zu können. Nach Specht et al. soll das Lernen dadurch zunehmend individueller werden \cite{specht}. Daraus ergeben sich natürlich auch Verbindungsoptionen mit anderen Technologiefeldern. Dazu gehören u.a. ortsbasierte und kontextsensitive Lerntechnologien mittels der Erfassung geotechnologischer Daten, Augmented Reality beim Einsatz von virtuellen Lernspielen sowie der Einsatz von künstlicher Intelligenz in interaktiven Lernanwendungen.

Der Trend geht somit klar in Richtung einer vernetzten Lernumgebung. Die in Abschnitt \ref{environment} dargestellten Eigenschaften des Cloud Computing verstärken diese Tendenz ganz klar. Somit ist nachfolgend Interessant, Einflussfaktoren für den erfolgreichen Einsatz dieser Technologie zu ermitteln. % Kapitel 3: Identifikation von Zukunftstrends
\section{Spezifikation von Einflussfaktoren}
\label{influencingfactors}  % Kapitel 4: Spezifikation von Einflussfaktoren
\section{Darstellung von Ausprägungen}
\label{manifestations}

Die zweite Phase der Szenarioanalyse bildet die Erstellung von Ausprägungen \cite{spath}. Hierfür werden Entwicklungspfade aus denen im vorhergehenden Kapitel definierten Schlüsselfaktoren gebildet.  % Kapitel 5: Erstellen von Ausprägungen
\section{Szenariokonstruktion}
\label{constructions}
Als dritte und wichtigste Phase der Szenarioanalyse werden nun Szenarien aus den vorher definierten Einflussfaktoren erstellt. Hierfür bietet es sich an, die vorher ermittelten Entwicklunsausprägungen der einzelnen Schlüsselfaktoren auf Widerspruchsfreiheit zu überprüfen . Dafür wird eine Konsistenzanalyse genutzt \cite{spath}. Auf dessen Grundlage werden danach zukunftsfähige Anwendungsszenarien für die Skola GmbH erstellt.

\subsection{Konsistenzanalyse}

In der in Abbildung \ref{fig:konsistenzanalyse} dargestellten Konsistenzanalyse erhält man eine Übersicht über die Beziehung zwischen den einzelnen Ausprägungen. Der Übersichtlichkeit halber wurden die einzelnen Ausprägungen abgekürzt. Die Erklärungen finden sich in Abschnitt \ref{manifestations}.

\begin{figure}
	\centering
	\includegraphics[width=\linewidth]{images/konsistenzanalyse}
	\caption[Caption for parameters]{Konsistenzanalyse}
	\label{fig:konsistenzanalyse}
\end{figure}

Für die Konsistenzanalyse wurde eine Skala zwischen -2 (starke Inkonsistenz) und 2 (konsistent und verstärkend) gewählt. Faktorenausprägungen, die eine starke Inkonsistenz aufweisen, können nicht zusammen in einem Szenario aufgeboten werden \cite{spath}. Andersrum können aus verstärkenden Ausprägungen Cluster gebildet werden, woraus sich Szenarien abbilden lassen.

Aus der Analyse der Schlüsselfaktoren auf Widerspruchsfreiheit lässt sich sehr gut erkennen, dass sich die jeweiligen Best-Case Ausprägungen zumeist verstärken. Besonders lässt sich ein positives Cluster aus den Ausprägungen \textit{Stabiles Netzwerk} (3a), \textit{mitwachsende Cloud} (4a), \textit{zuverlässige Cloud} (6a) und \textit{unabhängige Cloud} zusammenfassen. Andere positive Ausprägungen lassen sich zudem hinzuaddieren, sodass sie weiterhin verstärkend wirken. Im Gegensatz dazu verstärken sich die negativen Ausprägungen der Schlüsselfaktoren, wodurch sich ein entsprechendes negatives Cluster bilden lässt.

Aus dieser Betrachtung lässt sich erschließen, dass sich die Vorteile des Cloud Computing generell verstärkend aufeinander auswirken. Ebenso unterstützen sich die Risiken gegenseitig in ihrer Ausprägung. Diese Erkenntnis kann nun zur Hilfe gezogen zu werden, um drei mögliche Zukunftsszenarien für das Cloud Computing im Bildungsbreich zu erstellen.

\subsection{Szenario 1 - Mobiles Lernen auf dem Vormarsch}

In diesem Zukunftsszenario haben die positiven Ausprägungen des Cloud Computing Einklang in den Lernalltag von Schülerinnen und Schülern gefunden. Mobiles Lernen ist kein Ideal mehr sondern ein allgemeiner Zustand. Lernende haben jederzeit und überall Zugang zu Materialien über das Internet. E-Learning gehört ebenso zum Alltag wie eigenverantwortliches Nacharbeiten zu Hause mit dem eigenen Smartphone oder Tablet. Den Schülerinnen und Schülern wird Platz für freie Entfaltung gegeben, klassischer Unterricht besteht nur noch geringfügig. Schulbücher in der Form wie man sie aus klassischen Lernprinzipien kennt existieren bestenfalls als Zusatzmaterial. Große Schulbuchverlage wie Cornelsen oder Westermann haben ihre Angebote längst ins Internet mithilfe von Cloud Services verlagert. Dadurch bieten sie flexible, aber auch zahlreiche Lernmaterialien für jeden an. Zudem sind die angebotenen Dienstleistungen jederzeit sicher und entsprechen sämtlichen Datenschutzrichtlinien. 

In Betrachtung der derzeitigen Entwicklung des Cloud Computing ist ein solches Szenario in den nächsten 10-15 Jahren denkbar. Der Gebrauch der Technologie wächst immer weiter an \cite{krcmar} und Firmen der deutschen Bildungsindustrie überdenken ihr Angebot hinsichtlich eines Wechsels zu cloudbasierten Diensten \cite{grella}.

\subsection{Szenario 2 - Blended Learning (Hybrid)}

Dieses Szenario sieht ein Mischmodell aus Online - und Offline-Angebot von Unterrichtsmaterialien vor. Dies entspricht der in Abschnitt \ref{futuretrends} vorgegebenen Definition des \textit{Blended Learning}. Es sollen Synergien beider Modelle genutzt werden, um Vorteile beider Seiten auszunutzen und Nachteile möglichst zu beseitigen. Das Szenario sieht vor, dass Schülerinnen und Schüler eigenständig Hausaufgaben oder Nachholarbeiten über das cloudbasierte Web Services vornehmen, zum Beispiel durch E-Learning Angebote. Weiterhin bestehend bleibt aber der eigentliche Präsenzunterricht. Unterstützt wird dieser durch passende Web Services, die den Unterricht didaktisch bereichern \cite{meinel}.

Dieses Szenario ist ebenfalls in den nächsten 10-15 Jahren als ziemlich realistisch einzuschätzen. Zwar bleibt das Vertrauen in bewehrte Unterrichtskonzepte, auch hervorgerufen durch die Angst vor Datenschutzverletzungen, jedoch steigt der Einsatz von Cloud Diensten rasant \cite{krcmar}. 

\subsection{Szenario 3 - Zu hohe Risiken}

Eher unwahrscheinlich ist das dritte Szenario, welches beschreibt, dass Cloud Computing gar keinen Einklang in die Bildungsbranche finden wird. In diesem Fall sind die abgeschätzten Risiken der Technologie zu groß, um ernsthaft Vertrauen in eine Verlagerung von Lernmaterialien in die Cloud zu gewinnen. Hier wird weiter bewährte, klassische Konzepte vertraut. 

Unrealistisch ist dies deshalb, weil die Vorteile des Cloud Computing bei weitem überwiegen. Zwar gibt es berechtigte Risiken, wie zum Beispiel die Datensicherheit oder die Störanfälligkeit der bereitgestellten Infrastrukturen, diese können jedoch durch geeignete Konzepte eingeschränkt werden.

 % Kapitel 6: Szenariokonstruktion
\section{Zusammenfassung und Handlungsempfehlungen}
\label{conclusion}
Aus den erarbeiten Zukunftsszenarien ergeben sich nun Handlungsempfehlungen für die fiktive Skola GmbH. Diese können dementsprechend auf die tatsächliche Situation in der Bildungsbranche angewendet werden.

Die in Abschnitt \ref{constructions} gezeigten Szenarien haben gezeigt, dass ein Einsatz von Cloud Services durch das Unternehmen durchaus sinnvoll ist (vgl. Szenario 1 und 2). Dies kann auf unterschiedlichem Wege passieren. Als Schulbuchhersteller kann die Skola GmbH auf bereits vorhandene Ressourcen in Form ihrer Schulbücher sowie Arbeitsmaterialien zurückgreifen und diese beispielsweise als eigenständiger Cloud Provider im Internet anbieten. Dadurch kann man auf die Zukunftstrends \textit{mobiles Lernen}, \textit{E-Learning} oder \textit{Blended Learning} aufspringen. Man kann hierbei in alternative Richtungen gehen. Zum Beispiel könnte man im Bezahlmodell oder in der Art der angebotenen Dienste variieren (vgl. Abschnitt \ref{environment}). Dazu ist zwingend technisches Know-How notwendig. 

Will man als eigenständiger Provider auftreten, muss man zudem die Risiken der Cloud Technologie berücksichtigen (vgl. Szenario 3). Eine Reduzierung der Probleme kann zum Beispiel durch eine Verlagerung dieser Dienste auf externe Cloud Provider vorgenommen werden, wie zum Beispiel \textit{Amazon Web Services} oder \textit{Google Cloud Platform}. Dadurch liegt ein Großteil der Verantwortung bei den externen Anbietern und man kann sich auf die Bereitstellung der Lernmaterialien konzentrieren.

Wichtig in diesem Kontext ist zudem die Problematik des Datenschutzes. Geht man zum Beispiel den Weg über externe Dienstleister, muss beachtet werden, dass diese gegenwärtigen Richtlinien entsprechen. Gerade bei ausländischen Anbietern ist dies momentan schwer oder gar nicht gewährleistet. Kritisch in diesem Zusammenhang ist zudem der vorhandene Bildungsförderalismus in der deutschen Lernindustrie.

Grundsätzlich lässt sich jedoch zusammenfassen, dass das Cloud Computing enorme Potentiale im Bildungsbereich bietet. Es ist zunehmend wichtig, die Vorteile der Technologie zu bündeln, um diese Potentiale ausschöpfen zu können. Essentiell ist hierbei ein bundesweit einseitiges, datenschutzrechtliches ansprechendes Konzept. % Kapitel 7: Zusammenfassung und Handlungsempfehlungen

\begin{thebibliography}{00}
\bibitem{cloudservices} H. Krcmar, J. M. Leimeister, A. Roßnagel, und A. Sunyaev, Hrsg., Cloud-Services aus der Geschäftsperspektive. Wiesbaden: Springer Gabler, 2016.
\end{thebibliography}

\end{document}
